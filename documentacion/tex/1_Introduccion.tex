\capitulo{1}{Introducción}

Realizar una visita turística a una localidad desconocida puede resultar todo un reto. Ya sea por desconocer cómo es el lugar, no saber el tiempo que implica recorrer la distancia entre los puntos de interés, como por no desenvolverse con soltura en la lengua hablada en dicho lugar o simplemente por la falta de tiempo a la hora de preparar la visita.

Realizando una sencilla búsqueda por Internet se podrían localizar los lugares más importantes de la localidad elegida para ser visitada. Si se invirtiesen unos minutos más, se podrían localizar distintos museos, monumentos o plazas de interés. Añadiendo un tiempo adicional, se podrán localizar restaurantes, bares, etc. que pudiesen estar en lugares cercanos a los sitios turísticos seleccionados. Y si se dedicasen unos instantes más, se podría calcular una ruta adecuada para poder visitar todos esos lugares en el tiempo que la persona, cabe llamarla turista, puede dedicar a la localidad elegida anteriormente.

Aunque parece un espacio de tiempo relativamente corto, la planificación de una visita a un lugar desconocido puede llevar una cantidad de tiempo elevada. En la actualidad, invertir un tiempo elevado en esta tarea puede resultar complicado y, por eso, el presente Trabajo de Fin de Máster plantea una solución adecuada a este problema tan común.

La aplicación planteada en este trabajo trata de evitar la necesaria búsqueda de lugares de interés plantando una ruta adecuada al tiempo puede dedicar al destino turístico seleccionado.



\section{Estructura del documento}

La estructura del presente documento será la siguiente:

\subsection{Memoria}

En la parte correspondiente a la memoria se encontrarán las siguientes secciones:

\begin{enumerate}
	\item \textbf{Introducción:} en el primer apartado de la memoria se describe el contexto en el que el proyecto ha sido llevado a cabo.
	\item \textbf{Objetivos del proyecto:} el siguiente apartado detalla los objetivos marcados a la hora de realizar este proyecto. Se distingue entre objetivos generales, técnicos y personales.
	\item \textbf{Conceptos teóricos:} a continuación, se realiza una descripción de los conceptos necesarios para la realización del proyecto.
	\item \textbf{Técnicas y herramientas:} en este apartado se detallan las herramientas usadas para la
realización de este proyecto.
	\item \textbf{Aspectos relevantes del desarrollo del proyecto:} en este punto se detallan los aspectos más importantes o interesantes del desarrollo del proyecto resumiendo la experiencia práctica del mismo.
	\item \textbf{Estado del arte:} antes de finalizar, se realizará una comparación con otros trabajos o proyectos relacionados con este.
	\item \textbf{Conclusiones y líneas de trabajo futuras:} por último, en el apartado de conclusiones, se incluirá un conjunto de ideas resultantes de la experiencia recabada durante el desarrollo de este proyecto.
\end{enumerate}


\subsection{Anexos}
Los anexos quedan divididos en las siguientes secciones:

\begin{itemize}
	\item \textbf{Plan de proyecto:}
	\item \textbf{Especificación de requisitos:}
	\item \textbf{Especificación de diseño:}
	\item \textbf{Documentación técnica de programación:}
	\item \textbf{Documentación de usuario:}
\end{itemize}