\capitulo{1}{Introducción}

\section{Estructura del documento}

La estructura del presente documento será la siguiente:

\subsection{Memoria}

En la parte correspondiente a la memoria se encontrarán las siguientes secciones:

\begin{enumerate}
	\item \textbf{Introducción:} se describe el contexto en el que el proyecto ha sido llevado a cabo.
	\item \textbf{Objetivos del proyecto:} detalla los objetivos marcados a la hora de realizar este proyecto. Se distingue entre objetivos generales, técnicos y personales.
	\item \textbf{Conceptos teóricos:} realiza una descripción de los conceptos necesarios para la realización del proyecto.
	\item \textbf{Técnicas y herramientas:} detalla las herramientas usadas para la
realización de este proyecto.
	\item \textbf{Aspectos relevantes del desarrollo del proyecto:} contiene los aspectos más importantes o interesantes del desarrollo del proyecto resumiendo la experiencia práctica del mismo.
	\item \textbf{Estado del arte:} realiza una comparación con otros trabajos o proyectos relacionados con este.
	\item \textbf{Conclusiones y líneas de trabajo futuras:} por último se incluirá un conjunto de ideas resultantes de la experiencia recabada durante el desarrollo de este proyecto.
\end{enumerate}


\subsection{Anexos}
Los anexos quedan divididos en las siguientes secciones:

\begin{itemize}
	\item \textbf{Anexo I Plan de proyecto software:} permite mostrar la planificación temporal, económica y legal del proyecto.
	\item \textbf{Anexo II Especificación de requisitos:} este anexo muestra la especificación de requisitos de la plataforma web.
	\item \textbf{Anexo III Especificación de diseño:} detalla la estructura de los datos, el prototipado de la aplicación y la arquitectura del sistema.
	\item \textbf{Anexo IV Documentación técnica de programación:} permite explicar cómo instalar el entorno virtual necesario para ejecutar el sistema.
	\item \textbf{Anexo V Documentación de usuario:} en el manual de usuario se detallan los aspectos fundamentales a la hora de hacer uso de la plataforma.
	
\end{itemize}