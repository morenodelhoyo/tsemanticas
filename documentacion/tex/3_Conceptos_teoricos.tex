\capitulo{3}{Conceptos teóricos}

Para poder validar los datos de trayectorias en crudo obtenidos de distintos usuarios se han de seguir una serie de pasos que permitan limpiarlos y transformarlos en información útil para el sistema. Esta serie de pasos será descrita en esta sección de la memoria.

\section{Limpieza de datos}

Debido a fallos hardware o software, pueden capturarse rutas que no correspondan con la realidad. Por ejemplo, el receptor de GPS del dispositivo usado para la recolección de datos puede dejar de recibir señal durante unos minutos y provocar una unión entre un punto ``x'' y otro ``y'' habiendo perdido datos relevantes en dicho intervalo de tiempo. Esto provocará una posible línea recta entre coordenadas que no deberían estar relacionadas entre sí haciendo que la ruta pierda información sobre los puntos recorridos. Para solventar estos posibles fallos, se hará una limpieza de los datos recibidos  en el servidor y de los ya almacenados en el mismo.

Para ello se calculará la mediana entre los puntos de una misma ruta y, aplicando, un porcentaje de margen de error, se validará que la distancia entre los puntos no varíe de forma significativa. En caso de detectarse algún posible problema se procederá a validar la ruta sin la coordenada tomada como errónea. En caso contrario, la ruta quedará invalidada por completo y no será usada por el sistema. 

\section{Enriquecimiento semántico}

Los datos obtenidos por el usuario solo incluyen posiciones geográficas tomadas en un instante determinado, es decir, solo se tienen datos como la latitud y longitud y el espacio temporal en el que han sido tomados. En esta etapa se incluye información adicional a estos datos.

\subsection{Segmentación de la trayectoria}

Lo primero a hacer es dividir la trayectoria seguida en ``movimiento`` y ``paradas''. Un periodo de movimiento es el tiempo en el que la persona se encuentra moviéndose entre dos ubicaciones. Un periodo de parada es el tiempo en el que la persona se encuentra viendo un lugar de interés.


\section{Anotación de trayectorias}

Una vez conocidas las paradas que ha realizado la persona, se pueden asociar los Puntos de Interés a cada parada y así conocer los lugares en los que el usuario ha dedicado su tiempo.

\section{Consultas por anotaciones}

En este momento es posible realizar peticiones al sistema según lugares.


\section{Descubrimiento de conocimiento}

Extracción de patrones generalistas  a partir de las consultas realizadas.